\documentclass[11pt, letterpaper]{article}

\usepackage[margin=1in]{geometry} % manage page dimentions
\usepackage[utf8]{inputenc} % utf-8 encoding
\usepackage[italian]{babel} % italian default text
\usepackage[hidelinks]{hyperref} % link references
\usepackage{bookmark} % 
\usepackage{import} % import other .tex files
\usepackage{amsmath} % math commands
\usepackage{amssymb} % math symbols
\usepackage{amsthm} % math environments
\usepackage{amsmath} % equation align
\usepackage{siunitx} % SI unit
\usepackage{booktabs} % tabular enhance
\usepackage{multirow} % dinamic tabular cell dimentions
\usepackage{longtable} % multi-page table
\usepackage[labelfont=bf, skip=.5em, font=small]{caption} % beautiful caption
\usepackage{subcaption} % subfigure
\usepackage{graphicx} % import graphics
\usepackage{fancyhdr} % custom page header and footer

% \showthe\textwidth

\graphicspath{{../assets/}} % base graphics path

\setlength{\parskip}{1em} % distance between paragraphs
\setlength{\parindent}{0em} % indentation at beginning of paragraph
\setlength{\headheight}{13.59999pt}

\numberwithin{equation}{section} % equation tag relative to section

\pagestyle{fancy}
\fancyhead[L]{\nouppercase{\leftmark}}
\fancyhead[R]{\textbf{Pag. \thepage}}
\fancyfoot{}

\title{Esperienza 3}
\date{9/12/2021}
\author{}

\begin{document}

\maketitle

\thispagestyle{empty}

\tableofcontents

\section{Obiettivo dell'esperienza}


\section{Strumenti e materiali}

\begin{itemize}
    \item Generatore di tensione AC
    \item Multimetro digitale (utilizzato come ohmetro)
    \item Oscilloscopio
    \item Cavi
    \item Breadboard
    \item Resistore
    \item Condensatore
\end{itemize}

\section{Circuito}

\section{Onda quadra}

\begin{equation}
    V = V_{0} \cdot e^{-t/RC}
\end{equation}

\[
    \ln(V) = \ln(V_{0} \cdot e^{-t/RC}) = \ln(V_{0}) - \frac{t}{RC}
\]

Quindi facendo un grafico semi-logaritmico si ottiene una funzione lineare

\begin{figure}[ht!]
    \includegraphics{onda_quadra_V(t)_pendenze.pdf}
    \caption{Rette massima e minima pendenza}
    \label{fig:onda quadra pendenze}
\end{figure}

La retta di massima pendenza passa per i punti \((-0.2, 7)\) e \((35, 0.6)\) mentre la retta di minima pendenza passa per i punti \((0.2, 7)\) e \((34, 0.5)\)

\begin{equation*}
    m_{max} = \frac{\ln(7/0.6)}{- 0.2 - 35} = - 0.06979
    \qquad
    m_{min} = \frac{\ln(7/0.5)}{0.2 - 34} = - 0.07808
\end{equation*}

\begin{align*}
    m_{best} &= \frac{m_{max} + m_{min}}{2} = 0.0739 \\
    \delta m &= \frac{m_{max} - m_{min}}{2} = 0.0041
\end{align*}

\begin{equation}
    m = - 0.074 \pm 0.004
\end{equation}

Essendo \(m = \dfrac{1}{RC}\) e conoscendo il valore di \(R = (1.874 \pm 0.004) \; \unit{k\Omega}\) misurato tramite il multimetro

\begin{align*}
    \varepsilon_R &= \frac{0.004}{1.874} = 0.0021 \approx 0.002 \\
    \varepsilon_m &= \frac{0.004}{0.074} = 0.054 \approx 0.05 \\
    \varepsilon_C &= \sqrt{\varepsilon_R^{2} + \varepsilon_m^{2}} = 0.050 % review
\end{align*}

\begin{equation}
    C = \frac{R}{m} = 25.32 \pm 1.27 \approx (25.3 \pm 1.3) \; \unit{nF}
\end{equation}

\subsection{Dati ed errori}

\subsection{Analisi dati}

\section{Onda sinusoidale}

\subsection{Dati ed errori}

\subsection{Analisi dati}

\section{Conclusioni}

\end{document}